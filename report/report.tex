\documentclass[11pt]{article}

\usepackage[paper=letterpaper,
           marginparwidth=1in,
           hmargin={1.1in,0in},
           vmargin={1in,1.5in},           
           includemp
           ]{geometry}
           
 \usepackage{longtable}
 \usepackage{graphicx}
 \usepackage{caption}
\usepackage{subcaption}
\usepackage[]{natbib}
\usepackage{latexsym,amsfonts,amssymb,amsmath,amsthm,mathrsfs}
\usepackage{bbm}
\usepackage{amsthm}
\newtheorem{definition}{Definition}
\usepackage{hyperref}
\usepackage{lscape}

\hypersetup{
    bookmarks=true,         % show bookmarks bar?
    unicode=false,          % non-Latin characters in Acrobat’s bookmarks
    pdftoolbar=true,        % show Acrobat’s toolbar?
    pdfmenubar=true,        % show Acrobat’s menu?
    pdffitwindow=false,     % window fit to page when opened
    pdfstartview={FitH},    % fits the width of the page to the window
    pdftitle={My title},    % title
    pdfauthor={Author},     % author
    pdfsubject={Subject},   % subject of the document
    pdfcreator={Creator},   % creator of the document
    pdfproducer={Producer}, % producer of the document
    pdfkeywords={keyword1} {key2} {key3}, % list of keywords
    pdfnewwindow=true,      % links in new window
    colorlinks=true,       % false: boxed links; true: colored links
    linkcolor=red,          % color of internal links (change box color with linkbordercolor)
    citecolor=black,        % color of links to bibliography
    filecolor=magenta,      % color of file links
    urlcolor=blue           % color of external links
}

\title{Dynamic matching procedures for causal inference in social networks}

\providecommand{\keywords}[1]{\textbf{\textit{Keywords ---}} #1}


\author{{\textsc{Imanol Arrieta Ibarra}} \\
%EndAName
 \and {\textsc{Thomas Palomares}} \\
%EndAName
}



\begin{document}

\maketitle


\begin{abstract}
We design a matching mechanism to elicit the elect of friendship formation on a group of observed traits.
\end{abstract}

\newpage

\section{Introduction}

A social network is a community of people with common interests. Such networks are not static but evolve over time, suh as friendships being created and dissolved over the life course. 

\section{Network Creation Model}

Similar to \cite{christakis2007spread}, we developed a network creation model which is ran in two stages. In a first stage, we allow each node to nominate a friend. In the second stage we compute the happiness of all the nodes.

For each node we create a set of candidates based in probability: $\text{logit}^{-1} \left(-3||(X_i,Z_i)-(X_j,Z_j)||\right)$. Then from the set of candidates, we pick one at random to be the nominated friend. After the edges have been added, we compute the happiness according to the following equation:
$$h_{i,t} = w_0 +w_1 h_{i,t-1} + w_2\sum\limits h_{j,t-1} + w_3 X_{i} + w_4 Z_{i}$$

\subsection{Limitations}

The assumption that the current happiness depends only of the previous happiness of the person and of their friends is a clear limitation of the model. It is also a very useful assumption allowing us to easiily compute the happiness without computing an expensive fixed point equilibrium at each step.


\bibliographystyle{chicago}
\bibliography{biblio}

\end{document}